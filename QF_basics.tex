\section{二次函数的基本}

在八年级学习过函数的定义:一般地,在一个变化过程中,如果有两个变量\(x\)与\(y\),并且对于\(x\)的每一个确定的值,\(y\)都有唯一确定的值与之对应,那么我们就说\(x\)是自变量,\(y\)是\(x\)的函数.
\par
当自变量\(x\)和因变量\(y\)有如下关系:
\(y=kx+b\)
则此时称\(y\)是\(x\)的一次函数. 特别地,当\(b=0\)时,\(y\)是\(x\)的正比例函数.
\par
本章学习另一种函数:二次函数. 首先从二次函数的基本性质入手,再运用性质解决不同的问题,学习本章不仅需要熟悉基本的知识,还要积累经验和技巧.


\subsection{二次函数的定义}

\begin{definition}
一般地,形如\(y=ax^2+bx+c\)(\(a,\ b,\ c\)是常数,\(a\ne0\))的函数, 叫做二次函数.
\par
其中\(x\)是自变量,\(y\)是\(x\)的函数,\(a,\ b,\ c\)分别表示函数解析式的二次项系数、一次项系数、常数项.
\end{definition}

根据课本的定义,可以得到二次函数必须具有的条件:
\begin{enumerate}
    \item 关系式为整式
    \item 自变量最高次数为2(注意系数\(a\ne0\)保证二次项存在)
\end{enumerate}

\subsection{二次函数的图像}

\subsubsection{画二次函数的图像}
\begin{table}[h]
\centering
\renewcommand{\arraystretch}{1.2} % 增加行高
\begin{tabular}{|c|*{7}{c|}} \hline
\( x \) & \(-3\) & \(-2\) & \(-1\) & \(0\) & \(1\) & \(2\) & \(3\) \\ \hline\hline
\( y \) & \(9\) & \(4\) & \(1\) & \(0\) & \(1\) & \(4\) & \(9\) \\ \hline
\end{tabular}
\caption{\( y = x^2 \) 的函数值表}
\end{table}

先描点,再用平滑的线连接,就能得到这个二次函数的图像

%%%%%%%%%%%示例函数图像%%%%%%%%%%%%%%%%%%%%%%%%%%%%%%%%%%%%%%%%%%%%%%%%%%%%%%%%%%%%%%%%%%%%%%%%%%%%%%%%%%%%%%%%%%%%%%%%%%%%%%%%%%%%%%%%%%%%%%%%%%%%%%%%%%%%%%%%%%%%%%%%%
\begin{figure}[h]
    \centering
    \begin{tikzpicture}[scale=0.7]
    \begin{axis}[
        axis lines = middle,       % 坐标轴居中
        > = Stealth,
        xlabel = $x$,             % x轴标签
        ylabel = $y$,             % y轴标签
        xmin = -5, xmax = 5,      % x轴范围
        ymin = -1, ymax = 7,      % y轴范围
        xtick = {-4,-3,...,4},    % x轴刻度
        ytick = {-1,...,1,2,3,4,5,6},  % y轴刻度
        legend pos = north west,  % 图例位置
    ]
    
    \addplot [
        domain = -5:6,
        samples = 100,
        thick
    ] {x^2};
    
    \node[anchor=west] at (axis cs:3,6.5) {\(y=x^2\)};
    
    \end{axis}
    \end{tikzpicture}
    \caption{}
    \label{fuc_1}
\end{figure}

类似的,按照步骤能画出不同的函数图像.
\par
\subsubsection{二次函数图像的性质}
通过图像可以看出二次函数\(y=x^2\)的图像是一条抛物线,实际上所有的二次函数图像都是抛物线\footnote{伽利略(Galileo Galilei)(1564-1642)通过实验发现抛射体运动轨迹是抛物线,将物理现象与二次函数图像明确对应,笛卡尔(René Descartes)(1596-1650)创立解析几何(1637年《几何学》)首次用坐标系证明抛物线对应\(y=ax²+bx+c\)形式的方程.},所以我们把二次函数\(y=ax^2+bx+c\)的图像也叫做抛物线\(y=ax^2+bx+c\).
\par
观察图\ref{fuc_1}还能发现二次函数\(y=x^2\)是一个轴对称图形,而\(x=0\)则是他的对称轴,实际上所有二次函数都是轴对称图形,所有二次函数都有它的\textbf{对称轴},而\textbf{二次函数与对称轴的交点}称为二次函数的\textbf{顶点},在图\ref{fuc_1}中,抛物线\(y=x^2\)的顶点是\((0,0)\),也就是原点.
\par
观察图\ref{fuc_1}中\(x\)和\(y\)的关系,在抛物线的左侧也就是\(x>1\)时,\(y\)随\(x\)的增大而增大,当\(x<1\)时,\(y\)随\(x\)的增大而减小.函数在某一范围内随着自变量的增大而减小的性质可以取名为增减性\footnote{在未来的学习中,会更加准确的了解这个性质的定义,在高中教材称为单调性}.
\par
现在的抛物线$y=x^2$中,二次项系数的值为1,如果改变二次项系数$a$图像会发生变化吗?下面再用\(y=\dfrac{1}{2}x,y=-x^2, y=-\dfrac{1}{2}x^2\)三个函数作为示例.列表描点连线作出三个函数的图像
\par
对比函数图像,得出当\(a>0\)时,抛物线~~的开口向上,当$a<0$时,抛物线的开口向下,\(a\)的绝对值($|a|$)越大,开口越小,$|a|$越小开口越大

%%%%%%%%%%%%%%%%%%%%%%%%%%接例题判断开口等性质
\begin{example}
    写出下列抛物线的开口方向、对称轴、顶点坐标.
\end{example}
