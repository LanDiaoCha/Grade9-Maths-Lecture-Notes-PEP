\section{二次函数的基本}

在八年级学习过函数的定义:一般地,在一个变化过程中,如果有两个变量\(x\)与\(y\),并且对于\(x\)的每一个确定的值,\(y\)都有唯一确定的值与之对应,那么我们就说\(x\)是自变量,\(y\)是\(x\)的函数.
\par
当自变量\(x\)和因变量\(y\)有如下关系:
\(y=kx+b\)
则此时称\(y\)是\(x\)的一次函数. 特别地,当\(b=0\)时,\(y\)是\(x\)的正比例函数.
\par
本章学习另一种函数:二次函数. 首先从二次函数的基本性质入手,再运用性质解决不同的问题,学习本章不仅需要熟悉基本的知识,还要积累经验和技巧.


\subsection{二次函数的定义}

\begin{definition}
一般地,形如\(y=ax^2+bx+c\)(\(a,\ b,\ c\)是常数,\(a\ne0\))的函数, 叫做二次函数.
\par
其中\(x\)是自变量,\(y\)是\(x\)的函数,\(a,\ b,\ c\)分别表示函数解析式的二次项系数、一次项系数、常数项.
\end{definition}

根据课本的定义,可以得到二次函数必须具有的条件:
\begin{enumerate}
    \item 关系式为整式
    \item 自变量最高次数为2(注意系数\(a\ne0\)保证二次项存在)
\end{enumerate}

\subsection{二次函数的图像}

先描点,再用平滑的线连接,就能得到这个二次函数的图像

%%%%%%%%%%%示例函数图像%%%%%%%%%%%%%%%%%%%%%%%%%%%%%%%%%%%%%%%%%%%%%%%%%%%%%%%%%%%%%%%%%%%%%%%%%%%%%%%%%%%%%%%%%%%%%%%%%%%%%%%%%%%%%%%%%%%%%%%%%%%%%%%%%%%%%%%%%%%%%%%%%

\begin{table}[h]
\centering
\renewcommand{\arraystretch}{1.2} % 增加行高
\begin{tabular}{|c|*{7}{c|}} \hline
\( x \) & \(-3\) & \(-2\) & \(-1\) & \(0\) & \(1\) & \(2\) & \(3\) \\ \hline\hline
\( y \) & \(9\) & \(4\) & \(1\) & \(0\) & \(1\) & \(4\) & \(9\) \\ \hline
\end{tabular}
\caption{\( y = x^2 \) 的函数值表}
\end{table}

\begin{figure}[h]
    \centering
    \begin{tikzpicture}[scale=0.5]
    \begin{axis}[
        axis lines = middle,       % 坐标轴居中
        > = Stealth,
        xlabel = $x$,             % x轴标签
        ylabel = $y$,             % y轴标签
        xmin = -5, xmax = 5,      % x轴范围
        ymin = -1, ymax = 7,      % y轴范围
        xtick = {-4,-3,...,4},    % x轴刻度
        ytick = {-1,...,1,2,3,4,5,6},  % y轴刻度
        legend pos = north west,  % 图例位置
    ]
    
    \addplot [
        domain = -5:6,
        samples = 100,
        thick
    ] {x^2};
    
    \node[anchor=west] at (axis cs:3,6.5) {\(y=x^2\)};
    
    \end{axis}
    \end{tikzpicture}
    \caption{}
    \label{fuc_1}
\end{figure}

类似的,按照步骤能画出不同的函数图像.
\par
通过图像可以看出二次函数\(y=x^2\)的图像是一条抛物线,实际上所有的二次函数图像都是抛物线\footnote{伽利略(Galileo Galilei)(1564-1642)通过实验发现抛射体运动轨迹是抛物线,将物理现象与二次函数图像明确对应,笛卡尔(René Descartes)(1596-1650)创立解析几何(1637年《几何学》)首次用坐标系证明抛物线对应\(y=ax²+bx+c\)形式的方程.},所以我们把二次函数\(y=ax^2+bx+c\)的图像也叫做抛物线\(y=ax^2+bx+c\).
\par
观察图\ref{fuc_1}还能发现二次函数\(y=x^2\)是一个轴对称图形,而\(x=0\)则是他的对称轴,实际上所有二次函数都是轴对称图形,所有二次函数都有它的\textbf{对称轴},而\textbf{二次函数与对称轴的交点}称为二次函数的\textbf{顶点},在图\ref{fuc_1}中,抛物线\(y=x^2\)的顶点是\((0,0)\),也就是原点.
\par
图\ref{fuc_1}中\(x\)和\(y\)的关系,在抛物线的左侧也就是\(x>1\)时,\(y\)随\(x\)的增大而增大,当\(x<1\)时,\(y\)随\(x\)的增大而减小.函数在某一范围内随着自变量的增大而减小的性质可以取名为增减性\footnote{在未来的学习中,会更加准确的了解这个性质的定义,在高中教材称为单调性}.
\par
现在的抛物线$y=x^2$中,二次项系数的值为1,如果改变二次项系数$a$图像会发生变化吗?如果用作\(y=\dfrac{1}{2}x,y=-x^2, y=-\dfrac{1}{2}x^2\)三个函数的图,得出当\(a>0\)时,抛物线的开口向上,当$a<0$时,抛物线的开口向下,\(a\)的绝对值($|a|$)越大,开口越小,$|a|$越小开口越大

\begin{conclusion}
总结性质如下:
\begin{enumerate}
    \item 二次函数的图像是\textbf{轴对称图形},并且有唯一的一条\textbf{对称轴}.  
          对称轴与抛物线的交点称为\textbf{顶点},例如 \(y = x^2\) 的顶点是 \((0, 0)\).
    \item 对于二次函数 \(y = ax^2 + bx + c\),当自变量 \(x\) 在不同区间变化时,函数值会呈现\textbf{增减性}: 
    \begin{enumerate}
              \item 在顶点左侧,\(y\) 随 \(x\) 的增大而减小;  
              \item 在顶点右侧,\(y\) 随 \(x\) 的增大而增大.
          \end{enumerate}
    \item 二次项系数 \(a\) 对抛物线形状的影响:  
          \begin{enumerate}
              \item 当 \(a > 0\) 时,抛物线开口向上;  
              \item 当 \(a < 0\) 时,抛物线开口向下;  
              \item \(|a|\) 越大,抛物线开口越窄;  
              \item \(|a|\) 越小,抛物线开口越宽.
          \end{enumerate}
\end{enumerate}
\end{conclusion}


%%%%%%%%%%%%%%%%%%%%%%%%%%接例题图像性质

\begin{wrapfigure}{r}{3cm}
\vspace{-1cm}
\begin{tikzpicture}[scale=0.7, >=Stealth, declare function={f(\x) = -1.15*(\x-1)^2 + 2;}]
                                            % 坐标轴设置
                                            \draw[->, thick] (-1.5,0) -- (3.5,0) node[below] {$x$}; % x轴
                                            \draw[->, thick] (0,-0.5) -- (0,3) node[left] {$y$}; % y轴 (高度4cm)
                                            
                                            % 横轴刻度 (范围-1~3,不显示0)
                                            \foreach \x in {-1,1,2,3} {
                                                \draw (\x,0.1) -- (\x,-0.1) node[below] {$\x$};
                                            }
                                            
                                            % 原点标记为O
                                            \draw (0,0) node[below right] {$O$};
                                            
                                            % 对称轴 (x=1 虚线)
                                            \draw[densely dashed] (1,0) -- (1,2.5) node[above] {$x=1$};
                                            
                                            % 绘制二次函数 (开口向下)
                                            \draw[thick, domain=-0.5:2.5, samples=100] plot (\x, {f(\x)});
                                            
                                            \end{tikzpicture}
                                            \caption{}
                                            \label{b1}
\end{wrapfigure}


\exas{
如图\ref{b1},已知二次函数\(y=ax^2+bx+c\ (a\ne0)\)的图像如下,有下列5个结论:
\begin{enumerate*}[label=\textcircled{\arabic*}, itemjoin={;}]
    \item \(abc > 0\)
    \item \(4a+2b+c > 0\)
    \item \((a+c)^2> b^2\)
    \item \(2c>3b\)
    \item \(a+b>m(am+b)\)(\( m\ne 1\)的实数)
\end{enumerate*}
. 其中正确的有(\hspace{3em})
\par
\begin{enumerate*}[label=\Alph*., itemjoin={\hspace{3em}}]
    \item 1个
    \item 2个
    \item 3个
    \item 4个
\end{enumerate*}
}{}
%%%%%%%%%%%%%%%%%%%%%%%%%%%%%%%%%%%%%%%%%%%%%%%%%%%%
\begin{wrapfigure}{r}{3cm}
\begin{tikzpicture}[scale=0.7, >=Stealth, 
                                                    declare function={f(\x) = -1.15*(\x-1)^2 + 2.5;}] % 调整抛物线形状
                                                
                                                % 坐标轴设置
                                                \draw[->, thick] (-1.5,0) -- (3.5,0) node[below] {$x$}; % x轴
                                                \draw[->, thick] (0,-0.5) -- (0,3) node[left] {$y$}; % y轴
                                                
                                                % 原点标记
                                                \draw (0,0) node[below right] {$O$};
                                                
                                                % 对称轴 (x=1 虚线)
                                                \draw[densely dashed] (1,-0.3) node[right] {$E$} -- (1,2.7) node[right] {$D$};
                                                
                                                % 关键点标注
                                                \filldraw (-0.6,0) circle (0pt) node[below left] {$A$};
                                                \filldraw (2.7,0) circle (0pt) node[below right] {$B$};
                                                \filldraw (0,1.35) circle (0pt) node[left] {$C$}; % C点y坐标由f(0)计算
                                                
                                                % 绘制抛物线
                                                \draw[thick, domain=-0.6:2.6, samples=100] plot (\x, {f(\x)});
                                                
                                                
                                                \end{tikzpicture}
                                            \caption{}
                                            \label{b2}
\end{wrapfigure}

\exas{
如图\ref{b2},抛物线 \( y = ax^2 + bx + c \) 与 \( x \) 轴交于点 \( A(-1,0) \),\( B(3,0) \),交 \( y \) 轴的正半轴于点 \( C \),对称轴交抛物线于点 \( D \),交 \( x \) 轴于点 \( E \),则下列结论:
\begin{enumerate*}[label=\textcircled{\arabic*}, itemjoin={;}]
    \item \( b + 2c > 0 \)
    \item \( a + b \geq am^2 + bm \)(\( m \) 为任意实数)
    \item 若点 \( P \) 为对称轴上的动点,则 \( |PB - PC| \) 有最大值,最大值为 \( \sqrt{c^2 + 9} \)
    \item 若 \( m \) 是方程 \( ax^2 + bx + c = 0 \) 的一个根,则一定有 \( b^2 - 4ac = (2am + b)^2 \) 成立.
\end{enumerate*}
其中正确的序号有( )
}{哀伤地空间暗示登记卡时代科技哈萨克交点哈就开始打开教案设计肯定会哀伤地空间暗示登记卡时代科技哈萨克交点哈就开始打开教案设计肯定会哀伤地空间暗示登记卡时代科技哈萨克交点哈就开始打开教案设计肯定会哀伤地空间暗示登记卡时代科技哈萨克交点哈就开始打开教案设计肯定会哀伤地空间暗示登记卡时代科技哈萨克交点哈就开始打开教案设计肯定会哀伤地空间暗示登记卡时代科技哈萨克交点哈就开始打开教案设计肯定会哀伤地空间暗示登记卡时代科技哈萨克交点哈就开始打开教案设计肯定会哀伤地空间暗示登记卡时代科技哈萨克交点哈就开始打开教案设计肯定会哀伤地空间暗示登记卡时代科技哈萨克交点哈就开始打开教案设计肯定会哀伤地空间暗示登记卡时代科技哈萨克交点哈就开始打开教案设计肯定会哀伤地空间暗示登记卡时代科技哈萨克交点哈就开始打开教案设计肯定会哀伤地空间暗示登记卡时代科技哈萨克交点哈就开始打开教案设计肯定会}


